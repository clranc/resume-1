\documentclass[a4paper,margin,line]{resume}
\usepackage[defblank]{paralist}
\usepackage{pdfpages}
\usepackage{anysize}
\usepackage[unicode]{hyperref}
\hypersetup{
	pdftitle={Russell Harmon's Resume},
	pdfauthor={Russell Harmon},
	pdfborder={0 0 0},
	unicode=true
}
\marginsize{0.375in}{1.875in}{0.375in}{0.375in}
\setdefaultitem{\footnotesize \textbullet}{}{}{}{}{}
\setdefaultleftmargin{0em}{}{}{}{}{}
\setdefaultenum{(a)}{(1)}{}{}{}{}
\newcommand{\rurl}[1]{\hfill {\footnotesize \url{#1}}}
\newcommand{\rdate}[1]{\hfill {\small #1}}
\renewcommand{\employer}[5]{\item[#1] - #2 \rdate{#3} \\* #4 \rurl{#5} \\*}
\begin{document}
\name{\Large Russell E. Harmon}
\begin{resume}
\section{\mysidestyle Contact \\ Information} \vspace{2mm}
	\begin{asparablank}
		\item Rochester Institute of Technology \hfill (585) 210-3330
		\item PO Box 92423 \hfill \href{http://rus.har.mn/}{http://rus.har.mn}
		\item Rochester, NY 14692 \hfill \href{mailto:russ@eatnumber1.com}{russ@eatnumber1.com}
	\end{asparablank}

%\section{\mysidestyle Objective}
%	I am looking for a three month internship to begin March. I'll need to be allowed to work from Rochester, NY.

\section{\mysidestyle Summary}
	I love to tinker. I enjoy spending my time working on personal projects. Some of the notable projects I have worked on include the Linux kernel and the Gentoo Linux distribution.  I have become very good at picking up new ideas quickly, and expanding what I already know with a direction and my own research.

\section{\mysidestyle Education}
	\begin{compactdesc}
		\item[Rochester Institute of Technology] - Rochester, NY \rdate{September 2006 - Present}
		\begin{compactitem} { \small
			\item Major: B.S./M.S. Computer Science
			\item M.S. GPA: 3.0
			\item Minor: Music, Concentration: Computer Engineering
			\item Expected graduation: February 2013
		} \end{compactitem}
	\end{compactdesc}

\section{\mysidestyle Experience}
	\begin{asparadesc}
		\employer{Google}{Mountain View, CA}{6/5/2012 - Present}{Chrome OS Kernel Engineering Intern}{http://google.com/}

		\small
		Currently working on the Kernel team for the \href{http://www.google.com/intl/en/chrome/devices/}{Chromebook}. My work centers around the creation of a unit test which analyzes the kernel-to-kernel latency of input events, and the associated frameworks to make this test possible. This work involves adding trace events into the Linux kernel, X input drivers, and the Chromium browser; then building tools to analyze traces generated by these events.
		\normalsize
		\\
		\employer{Microsoft}{Boston, MA}{6/6/2011 - 8/26/2011}{Software Development Engineering Intern}{http://micorosft.com/}

		\small
		Worked on the "Sustained Engineering" team on the \href{http://www.microsoft.com/windows/enterprise/solutions/virtualization/products/app-v.aspx}{Application Virtualization} product. The sustained engineering team maintains the released version of App-V. Written in \emph{C++}, I worked on an as of yet unreleased feature for the App-V and Office products.
		\normalsize
		\\
		\employer{Apple Inc}{Cupertino, CA}{6/1/2010 - 11/16/2010}{Engineering Intern}{http://apple.com/}

		\small
		Worked on the "Platform Kernel" team in the Core OS department at Apple. The majority of my work entailed work on \emph{libC}, and on \emph{XNU} (the Mac OS X kernel). My two major projects were \begin{inparaenum} \item migrating code out of the kernel into userspace to take advantage of \emph{ASLR} and \item creation of an automated testing suite to test Mac OS X's power management on Intel processors. \end{inparaenum} The \emph{ASLR} work involved a great deal of collaboration between several different teams within Apple.
		\normalsize
		\\
%		\employer{Gentoo Linux}{Gentoo Foundation}{August 2007 - August 2010}{Package Maintainer}{http://www.gentoo.org/}
%
%		\small
%		Maintainer of several ebuilds. An ebuild is a bash script which fetches, unpacks, compiles and installs a piece of software.
%		\normalsize
%		\\
		\employer{SafeNet Inc}{Belcamp, MD}{June 2008 - June 2010}{Engineering Intern}{http://www.safenet-inc.com/}

		\small
		Worked on the SafeNet Management Console (SMC) on a team of 7. SMC is a web application built on \emph{Java}, \emph{JBoss}, \emph{Hibernate} and \emph{JSF} which manages high speed network encryption devices that SafeNet manufactures, including top-secret devices. Responsibilities included \begin{inparaenum} \item creating requirements documents, \item implementing requirements (IPv6, file synchronization), \item testing and \item fixing defects. \end{inparaenum} Some of the specific tasks that were assigned while working there included transparent file synchronization between machines, implementing IPv6 support and implementing support for multiple SMC servers managing the same device (distributed devices).
		\normalsize
		\\
%		\employer{New York City Department of Education}{Bronx, NY}{September 2002 - June 2006}{IT Specialist}{http://schools.nyc.gov/}
%
%		\small
%		Worked at Lehman High School. The work there included \begin{inparaenum} \item setup and maintain the computer systems, \item repair and maintain the network, \item diagnose and eliminate viruses and malware on the school's network and computer systems and \item audit the network for malicious activity. \end{inparaenum}
%		\normalsize
%		\\
%		\employer{New York Sailing \& Yacht Club}{Bronx, NY}{June 2001 - August 2005}{Launch Operator}{http://www.startsailing.com/}
%		
%		\small
%		Worked at the New York Sailing \& Yacht Club as the launch operator. The work involved taxiing patrons to their boats moored in the harbor. While there, I also received a boating license in small craft operation.
%		\normalsize
	\end{asparadesc}

\section{\mysidestyle Technical Skills \& Certifications}
	\begin{compactdesc}
		\item[Fluent Languages] \begin{inparaenum} { \small
			\item C
			\item Java
			\item ZSH
%			\item BASH
			\item C++
			\item Javascript
			\item Python
		} \end{inparaenum}
		\item[Operating Systems] \begin{inparaenum} { \small
			\item Linux [programming \& administration]
			\item Mac OS X
			\item Windows
		} \end{inparaenum}
		\item[Server Administration] \begin{inparaenum} { \small
			\item LDAP
			\item Kerberos
			\item BIND
			\item DHCPD
			\item Radvd
			\item Apache
		} \end{inparaenum}
		\item[Networking] \begin{inparaenum} { \small
			\item Cisco Academy, with honors (terms 1 and 2)
%			\item TCP/UDP
			\item IPv6
			\item DNS
			\item DHCP
			\item IPSec
%			\item SSL
%			\item X.509
		} \end{inparaenum}
		\item[Tools] \begin{inparaenum} { \small
			\item Make
			\item Autotools
			\item CMake
			\item LLVM
			\item Regex
			\item \LaTeX
		} \end{inparaenum}
%		\item[Certifications] \begin{inparaenum} { \small
%			\item Cisco Academy, with honors (terms 1 and 2)
%			\item A+
%			\item MOUS (Word)
%		} \end{inparaenum}
	\end{compactdesc}

%\section{\mysidestyle Extracurricular Clubs \& Activities}
%	\begin{asparablank}
%		\item CSH ({\small Computer Science House}) - Network Administrator\rurl{http://www.csh.rit.edu/}
%		\item US FIRST Robotics Team 1230 - {\small Lead programmer}\rurl{http://www.usfirst.org/}
%		\item Rocking the Boat - {\small Built a boat for children from the South Bronx} \rurl{http://www.rockingtheboat.org/}
%		\item City Island Computer Services - {\small Age 10}
%	\end{asparablank}

%\section{\mysidestyle Honors \& Awards}
%	\begin{asparablank}
%		\item NYC Regional Robotics Award winner - {\small 27th place nationally}
%		\item Visual Basic programming award for academic excellence 2005
%%		\item First Place Sailing Award 2000, 2002
%	\end{asparablank}

\section{\mysidestyle Special Accomplishments}
	\begin{asparablank}
		\item Joined CSH ({\small Computer Science House}). While there, I was elected by my peers first as a network administrator, then as the director of all the network administrators.\rurl{http://www.csh.rit.edu/}
		\item At age 8, built my first computer. At age 10, worked my first job at City Island Computer Services. At age 12, attended my first programming class at Lehman College.
%		\item Designed and implemented an encryption algorithm using a combination of transposition and substitution.
	\end{asparablank}

\end{resume}
\end{document}

% vim:set ai
